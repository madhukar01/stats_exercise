\documentclass[12pt]{article}
\usepackage[a4paper,margin=1in]{geometry}
\usepackage{graphicx}
\graphicspath{ {./images/} }

\author{Madhukara S Holla}
\title{Statistics Exercise 1}
\date{25th September 2023}

\begin{document}
\maketitle
\newpage
\section*{Question 1.a}
\includegraphics[width=\linewidth]{graph1a}

\subsubsection*{Key Observation}

\subsubsection*{Normal Distribution - \(\mathcal{N}(0, 1)\)}
\begin{itemize}
    \item Has a narrow curve due to low variance, resulting in a high probability
    density around the mean (0).
    \item It has shorter tails - samples drawn from this distribution will be closer
    to the mean.
    \item Does not have long tails - chances of drawing extreme values are low.
\end{itemize}

\subsubsection*{Normal Distribution - \(\mathcal{N}(3, 5)\)}
In comparison with \(\mathcal{N}(0, 1)\),
\begin{itemize}
    \item Has a wider curve due to high variance, resulting in a lower probability
    density around the mean (3).
    \item It has longer tails - a significant number of samples drawn from this
    distribution can be away from the mean (due to high variance).
    \item Tails are slightly longer than \(\mathcal{N}(0, 1)\), but chances of
    drawing extreme values are still low.
\end{itemize}

\subsubsection*{Cauchy Distribution - \(Cauchy(-2, 2)\)}
\begin{itemize}
    \item Has a narrower curve when compared to \(\mathcal{N}(3, 5)\) and has
    longer tails.
    \item This distribution is not symmetric and does not have a mean or variance.
    \item Chances of drawing extreme values are higher when compared to Normal distribution.
\end{itemize}

\newpage
\section*{Question 1.c}
\subsubsection*{QQ plot for samples from \(\mathcal{N}(0, 1)\)}
\includegraphics*[width=\linewidth]{graph1c}
From the QQ plot,
\begin{itemize}
    \item We can observe that the samples drawn from \(\mathcal{N}(0, 1)\)
closely align with the theoretical quantiles of normal distribution.
    \item They form a straight line, indicating that the samples are normally
    distributed.
\end{itemize}

\newpage
\section*{Question 1.d}
\subsubsection*{QQ plot for samples from \(\mathcal{N}(3, 5)\)}
\includegraphics*[width=\linewidth]{graph1d}
From the QQ plot,
\begin{itemize}
    \item We can observe that the samples drawn from \(\mathcal{N}(3, 5)\)
closely align with the theoretical quantiles of normal distribution.
    \item We can see a few points away from the straight line in the plot.
    This is because of the high variance in the distribution.
    \item Ignoring a few outliers, the samples are closer to a straight line,
    indicating that the samples are normally distributed.
\end{itemize}

\newpage
\section*{Question 1.e}
\subsubsection*{QQ plot for samples from \(Cauchy(-2, 2)\)}
\includegraphics[width=\linewidth]{graph1e}
From the QQ plot,
\begin{itemize}
    \item We can observe that the samples drawn from \(Cauchy(-2, 2)\) do not form
    a straight line - indicating that the samples are not normally distributed.
    \item In addition, we can see multiple outliers in the plot - indicating a
    longer tail.
\end{itemize}
\end{document}
