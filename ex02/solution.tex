\documentclass[12pt]{article}
\usepackage[a4paper,margin=1in]{geometry}
\usepackage{graphicx}
\graphicspath{{./images}}

\author{Madhukara S Holla}
\title{Statistics Exercise 2}
\date{13th Octoner 2023}

\begin{document}
\maketitle
\newpage
\section*{Question 1}
Project members: Madhukara S Holla (Master of Science in Computer Science, 1st Year)
\\
Project description: NA

\newpage
\section*{Question 2}
\subsection*{2.a}
\subsubsection*{Anscombe's Dataset 1}
\begin{itemize}
    \item Observed pearson co-efficient: 0.8164
    \item Observed co-efficient of multiple determination: 0.6665
\end{itemize}
\includegraphics*[width=\linewidth]{graph2a1}

\newpage
\subsection*{2.a}
\subsubsection*{Anscombe's Dataset 2}
\begin{itemize}
    \item Observed pearson co-efficient: 0.8162
    \item Observed co-efficient of multiple determination: 0.6662
\end{itemize}
\includegraphics*[width=\linewidth]{graph2a2}

\newpage
\subsection*{2.a}
\subsubsection*{Anscombe's Dataset 3}
\begin{itemize}
    \item Observed pearson co-efficient: 0.8163
    \item Observed co-efficient of multiple determination: 0.6663
\end{itemize}
\includegraphics*[width=\linewidth]{graph2a3}

\newpage
\subsection*{2.a}
\subsubsection*{Anscombe's Dataset 4}
\begin{itemize}
    \item Observed pearson co-efficient: 0.8165
    \item Observed co-efficient of multiple determination: 0.6667
\end{itemize}
\includegraphics*[width=\linewidth]{graph2a4}

\newpage
\subsection*{2.b}
Lack of fit tests for Anscombe's datasets.
\\
Datasets grouped by x values (2 values per group)
\\[\baselineskip]
Null Hypothesis \(H_0\): A simple linear model is adequate to explain the systematic
variations in the data.
\\[\baselineskip]
Alternate Hypothesis \(H_a\): A linear model is not adequate and a nonlinear model
is required to capture the systematic variations in the data.

\subsubsection*{Anscombe's Dataset 1}
P value: 0.86 - Fail to reject \(H_0\).
\\
No significant lack of fit.The dataset appears to be a simple linear relationship,
and a linear regression model seems appropriate for this dataset.

\subsubsection*{Anscombe's Dataset 2}
P value: 0.03 - Reject \(H_0\) in favor of \(H_a\).
\\
Significant lack of fit. The data clearly follows a non-linear (quadratic)
relationship, indicating that the linear model does not capture all systematic
variations in the dataset.

\subsubsection*{Anscombe's Dataset 3}
P value: 0.83 - Fail to reject \(H_0\).
\\
The outlier is ignored when we group the data by x values in pairs of 2.
\\
No significant lack of fit. Since the influence of the outlier is ignored while
calculating lack of fit, the dataset appears to be a simple linear relationship.

\subsubsection*{Anscombe's Dataset 4}
P value: 0.04 - Reject \(H_0\) in favor of \(H_a\).
\\
The outlier is ignored when we group the data by x values in pairs of 2.
\\
Significant lack of fit. Test is not be appropriate due to the nature of the data.
If forced, likely a significant lack of fit.
\\[\baselineskip]
The lack of fit test assumes that there's some variation in the independent variable (x)
that corresponds to variation in the dependent variable (y).
In Dataset 4, for all but one observation, there's no variation in x.
This goes against the fundamental premise of regression that
we're trying to understand how y changes as x changes.

\newpage
\subsection*{2.c}
\begin{itemize}
    \item Patterns in residual plots can indicate non-linearity and outliers,
    helping us to identify problems with the model.
    \item Lack of fit tests provide a formal statistical test to validate
    the model assumptions.
    \item But the lack of fit tests require replicate observations in the data
    which may not be available, or it may be inappropriate to run on datasets
    like Dataset 4.
    \item Both pearson coefficient and co-efficient of multiple determination
    do not clearly indicate the goodness of fit of the model. They just indicate
    the strength of the linear relationship and proportion of variance explained
    by the model respectively.
\end{itemize}
In conclusion, we need to use multiple methods such as visualization, lack of fit
tests, and co-efficient determinations to validate the model assumptions.
\end{document}
